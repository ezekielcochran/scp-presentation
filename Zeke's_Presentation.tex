\documentclass[x11names]{beamer}
\usetheme{Ilmenau}
\usepackage{wrapfig}
\usepackage{nicefrac}
\usepackage[T1]{fontenc}
\usepackage[utf8]{inputenc}
\usecolortheme[named=SteelBlue4]{structure}
\usepackage{amsmath}
\usepackage{amssymb}
\usepackage{amsfonts}
\usepackage{amsthm}
\usepackage{mathrsfs}
\usepackage{mathtools}
\newcommand{\N}{\mathbb{N}}
\newcommand{\p}{\mathcal{P}}
\newcommand{\s}{\mathcal{S}}
\newcommand{\Ll}{\mathscr{L}}
\newcommand{\SCP}{sequentially congruent partition}
\newcommand{\tail}{\operatorname{Tail}}
\newcommand{\lcm}{\operatorname{lcm}}
\newtheorem{prop}{Proposition}
\definecolor{shadedcell}{RGB}{71, 101, 125}
\definecolor{shadedcelllight}{RGB}{163, 178, 190}
\usepackage{graphicx}
\usepackage{young}
\usepackage{ytableau}
\setbeamercolor{block title example}{bg=SteelBlue4}
\usepackage{tikz}
\usetikzlibrary{patterns,snakes}

\makeatletter
\newenvironment<>{proofs}[1][\proofname]{%
    \par
    \def\insertproofname{#1\@addpunct{.}}%
    \usebeamertemplate{proof begin}#2}
  {\usebeamertemplate{proof end}}
\makeatother


\title{Sequentially Congruent Partitions}
\subtitle{REU 2022}
\author{Ezekiel Cochran, Emma Harrell, and Samuel Saunders}
\institute{The University of Texas at Tyler}
\date{September 17, 2022}

\begin{document}

\begin{frame}{}
    \titlepage
    \vspace{-15mm}
    % \begin{center}
    \begin{tikzpicture}
    \node at (5,0)
        {\includegraphics[scale=0.05]{NSF.png}};
    \node at (-4,0)
        {\includegraphics[scale=0.18]{utt-logo.png}};
    \end{tikzpicture}
    % \end{center}
\end{frame}




\section{Introduction}
\begin{frame}{What is a Partition?}
    A \emph{partition} $\lambda$ of a positive integer $n$ is a sequence $\lambda=(\lambda_1, \lambda_2, \dots, \lambda_r)$ of positive integers where $\lambda_1\geq \lambda_2\geq\dots\geq\lambda_r$ and $\lambda_1+\lambda_2+\dots+\lambda_r=n$.

    \vspace{3mm}\pause
    
    There are seven partitions of 5:
    \[(5)\;\;\;\;\;\;\;\;\;\;(4,1)\;\;\;\;\;\;\;\;(3,2)\;\;\;\;\;\;\;\;\;(3,1,1)\]
    \[(2,2,1)\;\;\;\;\;\;(2,1,1,1)\;\;\;\;\;\;(1,1,1,1,1)\]

\pause
\vspace{3mm} 

We use $\ell(\lambda)$ to denote the length of $\lambda$, and $|\lambda|$ to denote the size of $\lambda$.
\end{frame}

\begin{frame}{Frequency Notation}
Sometimes, we represent a partition $\lambda$ as $\lambda = \langle 1^{f_1}, 2^{f_2}, \dots \rangle$, where $f_i$ is the number of times part $i$ occurs in the partition. For example, \[(7,6,3,3,1)=\langle 1^1, 3^2, 6^1, 7^1 \rangle.\] 
\end{frame}

\begin{frame}{Young Diagrams}

\begin{itemize}
    \item graphical representation of a partition $\lambda$
    \pause
    \item $i$th row has $\lambda_i$ nodes or boxes
    \onslide<4,5>{
    \item conjugate: rows become columns and vice versa }
\end{itemize}

    \vspace{1mm}

    \onslide<3,4,5>{    Young diagram for $(7, 5, 5, 4, 1)$}\;\;\;\;\;\;\;\;\;\;\;\;\;\;\;\;\;\;\;\; \onslide<5>{and its conjugate}
    \vspace{1mm}
    \begin{columns}
    \column{0.5\textwidth}
        \centering
        \onslide<3,4,5>{ \begin{Young}
    & & & & & & \cr
& & & & \cr
& & & & \cr
& & & \cr
\cr
\end{Young}}
    \column{0.5\textwidth}
        \centering
        \onslide<5>{ \begin{Young}
    & & & &  \cr
& & &  \cr
& & &  \cr
& & & \cr
& & \cr
\cr
\cr
\end{Young}}
    \end{columns}

\end{frame}

\begin{frame}{Sequentially Congruent Partitions}
    A partition $\lambda=(\lambda_1,\lambda_2,...,\lambda_r)$ is \emph{sequentially congruent} if \begin{enumerate}
    \item $\lambda_i \equiv  \lambda_{i + 1} \mod i$ for all $1 \leq i < r$
    \item $\lambda_r \equiv 0 \mod r$
\end{enumerate}
    \pause
    \begin{example}
    $(21, 14, 14, 8)$ is sequentially congruent since 
    \begin{align*}
        &4\mid8\\
        &3\mid (14-8)\\
        &2\mid (14-14)\\
        \shortintertext{and}&1\mid (21-14) 
    \end{align*}
    \end{example}
    
\end{frame}

\begin{frame}{Previous Work}


    \begin{itemize}
        \item Defined in 2019 by Schneider and Schneider
        
        \begin{itemize}
            \item  $\p_n$ is the set of partitions of $n$
            \item  $\s_{\text{lg}=n}$ is the set of sequentially congruent partitions with largest part $n$
            \item Two distinct bijections found between $\p_n$ and $\s_{\text{lg}=n}$
        \end{itemize}
    \end{itemize}
    
        
        \vspace{3mm}\pause
        
    $$\sum_{n = 1}^{\infty} p(n)q^n = \prod_{n = 1}^{\infty} \frac{1}{1-q^n}.$$
    
    
\end{frame}

\begin{frame}{Previous Work}
\begin{itemize}
        \item In 2020, Schneider, Sellers, and Wagner showed that sequentially congruent partitions of size $n$ are in bijection with partitions whose parts are squares of size $n$ so that 
        \vspace{3mm}
        \pause
        \ $$\sum_{n = 1}^{\infty} p(\s, n)q^n=\sum_{n = 1}^{\infty} p_{\square}(n)q^n = \prod_{n = 1}^{\infty} \frac{1}{1-q^{n^2}}.$$
        where $p_\square(n)$ denotes the number of partitions whose parts are perfect squares.
        \end{itemize}
        
\end{frame}

\begin{frame}{Schneider--Schneider Bijections: $\pi$}
The map $\pi: \p_n\to\s_{\text{lg}=n}$ is defined by $\pi((\lambda_1, \lambda_2, \dots,\lambda_r))=(\lambda_1', \lambda_2', \dots,\lambda_r')$, where $$\lambda_i'=i\lambda_i+\sum_{j=i+1}^r\lambda_j.$$

\pause

\begin{example}
\begin{align*}
    \pi((6,3,3,2)) &= (1(6)+3+3+2, 2(3)+3+2, 3(3)+2, 4(2))\\
    &= (14, 11, 11, 8).
\end{align*}

\end{example}
\end{frame}

\begin{frame}{Schneider--Schneider Bijections: $\sigma$}
The map $\sigma:\s_{\text{lg}=n}\to\p_n$ is defined by  
\begin{align*}
\sigma &((\phi_1, \phi_2, \dots,\phi_r)) = \\
&\left\langle1^{\phi_1-\phi_2}, 2^{\nicefrac{(\phi_2-\phi_3)}{2}},3^{\nicefrac{(\phi_3-\phi_4)}{3}}, \dots, (r-1)^{\nicefrac{(\phi_{r-1}-\phi_r)}{(r-1)}}, r^{\nicefrac{\phi_r}{r}}\right\rangle
\end{align*}


\pause
\begin{example}\begin{align*}
    \sigma((13,13,11,8))&=\left<1^0,2^1,3^1,4^2\right>\\
    &=(4,4,3,2).
\end{align*} \end{example}
\end{frame}


\begin{frame}{Schneider--Schneider Bijections}
\begin{itemize}
    \item $\sigma\circ\pi$ is equivalent to conjugation
\end{itemize} 
\vspace{1mm}
\pause
\begin{corollary}
The conjugate of $\lambda=(\lambda_1,\lambda_2, \dots, \lambda_r)$ is given by $\left<1^{\lambda_1-\lambda_2}, 2^{\lambda_2-\lambda_3}, \dots,(r-1)^{\lambda_{r-1}-\lambda_r}, r^{\lambda_r}\right>$.
\end{corollary}
\pause
\vspace{2mm}
\begin{itemize}
    \item $\pi\circ\sigma$ is equivalent to an analogue to conjugation specific to sequentially congruent partitions
\end{itemize}
\end{frame}

\section{C-notation}

% \begin{frame}{Defining the Operation $\star$}
%     For two partitions, $\lambda, \gamma$, we define $\lambda\star\gamma$ to be the partition given by $(\lambda\star\gamma)_i=\lambda_i+\gamma_i$ where if $i>\ell(\lambda)$ we define $\lambda_i=0$.\\
    
%     \vspace{3mm}
    
%     We also define $c\lambda=\underbrace{\lambda\star\lambda\star\dots\star\lambda}_\text{$c$ times}$. For example, $3(5,4,2)=(5,4,2)\star(5,4,2)\star(5,4,2)=(3\cdot5,3\cdot4,3\cdot2)=(15,12,6)$.
    
%     \vspace{3mm}
    
%     Note that $(\s, \star)$ defines a commutative monoid (with the cancellative property) since if $\lambda \text{ and } \gamma$ are sequentially congruent partitions then $\lambda_k-\lambda_{k+1}=kn,  \gamma_k-\gamma_{k+1}=km$ with $m, n\in \mathbb{Z}_{\geq 0}$ so that $k\mid[(\lambda\star\gamma)_i-(\lambda\star\gamma)_{i+1}]$.
% \end{frame}

\begin{frame}{The "Primes"}
    Given any \SCP, you can add any $n \in \N$ to the first $n$ terms, to get another \SCP. \\
    \pause
    In fact, any \SCP\ can be "built" this way:
    \begin{align*}
        &c_1(1) \star \\
        &c_2(2, 2) \star \\
        &c_3(3, 3, 3) \star \\
        &\vdots \\
        &c_r(\underbrace{r, \dots, r}_{r \text{ times}})
    \end{align*}
    Where $c_i \in \N \cup \{ 0 \}$ for all $i$, finitely many nonzero \\
    The $\star$ operation represents left-justified partwise addition 
\end{frame}

\begin{frame}{C-Notation} %This is the frame that VSCode does not like
    \begin{theorem} A partition is sequentially congruent if and only if it can be written as \[\lambda = c_1(1) \star c_2(2,2) \star c_3(3,3,3) \star \dots \star c_r (r, \dots,r) \] where $c_1$ through $c_r$ are non-negative integers.
    \end{theorem}
    \pause
    Notation: We write this $\lambda$ as $[c_1, \cdots, c_r]$ \\
    \vspace{10mm}
    \pause
    Our largest part is $\sum_{i = 1}^{r} i c_i$, and the size is $\sum_{i = 1}^r i^2 c_i$.
\end{frame}

\begin{frame}{C-Notation and Young Diagrams}

\ytableausetup{boxsize = 1em, centertableaux}
\begin{ytableau}
\none[\cdots] & & & & & & & \none[\cdots] & & & & & &\none[\cdots] & & & & & \none[\cdots] & & & & \none[\cdots] & & & \none[\cdots] & \\
\none & & & & & & & \none & & & & & & \none & & & & & \none & & & & \none & & & \none\\
\none & & & & & & & \none & & & & & & \none & & & & & \none & & & & \none\\
\none & & & & & & & \none & & & & & & \none & & & & & \none\\
\none & & & & & & & \none & & & & & & \none \\
\none & & & & & &
\end{ytableau}

\begin{itemize}
    \item The $i\times i$ square occurs $c_i$ times.
\end{itemize}
    
\end{frame}

\begin{frame}{Bijection with Partitions into Squares}

\only<1>{For example, consider the partition $(16, 13, 11, 5, 5)$.}
\only<2>{For example, consider the partition $(16, 13, 11, 5, 5) =[3,1,2,0,1]$.}
\only<3>{The sequentially congruent partition $[3,1,2,0,1]$ is in bijection with the partition into squares $\langle (1^2)^3, (2^2)^1, (3^2)^2, (5^2)^1 \rangle $.}

\vspace{2mm}

% \ytableausetup
% {mathmode, boxsize=1em, centertableaux}
% \begin{ytableau}
%  *(white)  & & & & & \uncover<2,3>{\blacksquare} & \uncover<2,3>{\blacksquare} & \uncover<2,3>{\blacksquare}& & & & \uncover<2,3>{\blacksquare} & \uncover<2,3>{\blacksquare} & & \uncover<2,3>{\blacksquare}& \\
%  *(white)  & & & & & \uncover<2,3>{\blacksquare} & \uncover<2,3>{\blacksquare} & \uncover<2,3>{\blacksquare}& & & & \uncover<2,3>{\blacksquare} & \uncover<2,3>{\blacksquare}\\
%  *(white)  & & & & & \uncover<2,3>{\blacksquare} & \uncover<2,3>{\blacksquare} & \uncover<2,3>{\blacksquare}& & & \\
%  & & & & \\
%  & & & &
% \end{ytableau}

\only<1>{

\ytableausetup
{mathmode, boxsize=1em, centertableaux}
\begin{ytableau}
    *(white)  & & & & &  &  & & & & &  &  & & & \\
    *(white)  & & & & &  &  & & & & &  & \\
    *(white)  & & & & &  &  & & & & \\
    & & & & \\
    & & & &
\end{ytableau}
}

\uncover<2, 3>{

\ytableausetup
{mathmode, boxsize=1em, centertableaux}
\begin{ytableau}
    *(shadedcelllight)  & *(shadedcelllight) &*(shadedcelllight) &*(shadedcelllight) &*(shadedcelllight) & *(shadedcell) & *(shadedcell) & *(shadedcell)&*(shadedcelllight) &*(shadedcelllight) &*(shadedcelllight) & *(shadedcell) & *(shadedcell) &*(shadedcelllight) & *(shadedcell)&*(shadedcelllight) \\
    *(shadedcelllight)  &*(shadedcelllight) &*(shadedcelllight) &*(shadedcelllight) &*(shadedcelllight) & *(shadedcell) & *(shadedcell) & *(shadedcell)&*(shadedcelllight) &*(shadedcelllight) &*(shadedcelllight) & *(shadedcell) & *(shadedcell)\\
    *(shadedcelllight)  &*(shadedcelllight) &*(shadedcelllight) &*(shadedcelllight) &*(shadedcelllight) & *(shadedcell) & *(shadedcell) & *(shadedcell)&*(shadedcelllight) &*(shadedcelllight) &*(shadedcelllight) \\
    *(shadedcelllight) &*(shadedcelllight) &*(shadedcelllight) &*(shadedcelllight) &*(shadedcelllight) \\
    *(shadedcelllight) &*(shadedcelllight) & *(shadedcelllight)&*(shadedcelllight) &*(shadedcelllight)
\end{ytableau}

}

\uncover<3>{

\vspace{2mm}

\ytableausetup
{mathmode, boxsize=1em, centertableaux}
\begin{ytableau} 
*(shadedcelllight) &*(shadedcelllight) &*(shadedcelllight) &*(shadedcelllight) &*(shadedcelllight) &*(shadedcelllight) &*(shadedcelllight) &*(shadedcelllight) &*(shadedcelllight) &*(shadedcelllight) &*(shadedcelllight) &*(shadedcelllight) &*(shadedcelllight) &*(shadedcelllight) &*(shadedcelllight) &*(shadedcelllight) &*(shadedcelllight) &*(shadedcelllight) &*(shadedcelllight) &*(shadedcelllight) &*(shadedcelllight) &*(shadedcelllight) &*(shadedcelllight) &*(shadedcelllight) &*(shadedcelllight)\\
*(shadedcell) &*(shadedcell) &*(shadedcell) & *(shadedcell)& *(shadedcell)& *(shadedcell)& *(shadedcell)&*(shadedcell) & *(shadedcell)\\
*(shadedcelllight) &*(shadedcelllight) &*(shadedcelllight) &*(shadedcelllight) &*(shadedcelllight) &*(shadedcelllight) &*(shadedcelllight) &*(shadedcelllight) &*(shadedcelllight) \\
*(shadedcell) & *(shadedcell) &*(shadedcell) &*(shadedcell)\\
*(shadedcelllight)\\
*(shadedcell)\\ 
*(shadedcelllight)\\
\end{ytableau}}

\end{frame}

\section{Partition Bijections}

\begin{frame}{Bijection with Partitions into Squares}
    Once we are able to think about \SCP s in their c-notation, the bijection into squares is immediately apparent.
    
    % \pause
    \vspace{3mm}
    
    Call this bijection $\psi$, so that $$\psi \left( [c_1,c_2,\dots,c_r] \right) = \left< \left( 1^2 \right)^{c_1}, \left( 2^2 \right)^{c_2},\dots, \left( r^2 \right)^{c_r} \right>.$$
    
    \pause
    \vspace{3mm}
    This bijection is also easily visible from the Young diagrams. \\
    Thus the generating function for sequentially congruent partitions is $$\sum_{n = 1}^{\infty} p(\s,n)q^n = \prod_{n = 1}^{\infty} \frac{1}{1-q^{n^2}}.$$
\end{frame}

% Added this whitespace to keep line numbers consistant (There was commented code here)










\begin{frame}{$\pi$ and $\sigma$}
    % While most partition bijections map size to size, some are maps between other characteristics. In their 2019 paper, Schneider--Schneider defined two partition bijections between the set of sequentially congruent partitions with largest part $n$, $\s_{lg=n}$, and the set of partitions of size $n$, $\p_n$. 
    
    \begin{itemize}
        \item Most bijections map size to size \\
        \pause
        \item The set of \SCP s with largest part $n$, $\s_{lg = n}$ \\
        \pause
        \item The set of partitions of size $n$, $\p_n$ \\
        \pause
        \item Schneider-Schneider found two bijections between $\s_{lg = n}$ and $\p_n$\\
        \item Much simpler to write in c-notation
    \end{itemize}
    
    \vspace{3mm}
    
    % While these bijections, $\pi$ and $\sigma$, were somewhat complicated, using c-notation, we are able to describe the same bijections in a much simpler way. 
\end{frame}

\end{document}